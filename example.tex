\def\stit{\pg;\tit}
\def\ssec{\pg;\sec}
\ttline=-2

\load[gtex,modern]

\global\typosize[24/27]
\global\everytt={\typosize[24/27]}

\tit Showing off \GTeX

\def\remotemethod{(@ (guix git-download) git-fetch)}
\let\uriwrapper\useit
\def\inkscapeactions{select:g3122; delete}
\guix{guix-logo}{\svgrender{
  (file-append
        \remotefile{((@ (guix git-download) git-reference)
                        (url "git://git.savannah.gnu.org/guix/guix-artwork.git")
                        (commit "693f77b9b5e7f70b0e7cc7cba8b58df6c30b3ab2"))
                        }{Q+E5ED5FPWQdPnKt+dKTbfvABhbIfW8ZR3+1UdNwG9I=}
        "/logo/Guix.svg")
}}

\def\insapdownscale{0.8}
\insappic{guix-logo}

\stit Line-by-line evaluation

\pg;
\sec Node
\vfil
{
\def\printer#1{console.log(#1)}
\def\evaluator{node}
\showline{1 + 1}
\showline{(3/5) + 1}
\showline{[1,2,3,4,5].map(x => x + 1)}
}

\ssec Guile

\vfil
\showline{(+ 1 1)}
\showline{(+ (/ 3 5) 1)}
\showline{(map 1+ '(1 2 3 4 5))}
\pg+\showline{"Strings also work"}

\ssec Special syntax

«node
x = { y : 2 }
console.log(x['y'] + 5)
»

\begtt
 «node
 x = { y : 2 }
 console.log(x['y'] + 5)
 »
\endtt

\ssec Node

You can still include standard text:

* even \OpTeX{} lists
* many of them
\vskip50pt

«node
x = { y : 2 }
console.log(x['y'] + 5)
»

\ssec Python 3

«python/python3
def foo():
  print(10 % 3)
  return 0 % 2
  another

foo()
»

\ssec Python 2

«python2
print 10 % 3
»

\ssec Complex Python

% https://matplotlib.org/stable/gallery/

\beggt{matplotlib}{python, python-matplotlib | python3}
import matplotlib.pyplot as plt
import sys

data = {'apple': 10, 'orange': 15, 'lemon': 5, 'lime': 20}
names = list(data.keys())
values = list(data.values())

fig, axs = plt.subplots(1, 3, figsize=(9, 3), sharey=True)
axs[0].bar(names, values)
axs[1].scatter(names, values)
axs[2].plot(names, values)
fig.suptitle('Categorical Plotting')
plt.savefig(sys.stdout.buffer, format = 'png', dpi = 1200, transparent = True)
\endgt

\insappic{matplotlib}%

\ssec SBCL

«sbcl --script
(defun foo (i)
  (loop for x to i do (format t "Hello ~a~%" x)))

(foo 5)
»

\ssec Hubble

\rinspic{hubble}{https://upload.wikimedia.org/wikipedia/commons/thumb/2/2f/Hubble_ultra_deep_field.jpg/1200px-Hubble_ultra_deep_field.jpg}{XI2MgQY50Y3iPsfaQCvoEY/siwhl+LB1Vnjy0RkKJAg=}

\ssec Win 7

\rinspic{im}{https://i.stack.imgur.com/LjGip.png}{sB5X/UMwqmWYLiqiN8jiQDwfN13QQ5cII8BA/w7ximo=}

\ssec Sequent (SVG)

\rinssvg{svg}{https://wikimedia.org/api/rest_v1/media/math/render/svg/e0c5b66a8fed889a35d2ea5ff95f0c604351346d}{3yC9Cu5w2uIKfAy8LEZERNCVDG6zkztJ3rc7PaqnSeA=}

\ssec GraphViz 1

\graphviz{
  digraph {
    graph [dpi=600];
    bgcolor = transparent
    a -> b
    b -> c [color=red]
    subgraph cluster_top_floor {
      a -> c
    }
  }
}

\ssec GraphViz 2

\graphviz{
  % http://magjac.com/graphviz-visual-editor/
  graph {
    bgcolor=lightblue
    label=Home
    subgraph cluster_ground_floor {
      bgcolor=lightgreen
      label="Ground Floor"
      Lounge
      Kitchen
    }
    subgraph cluster_top_floor {
      bgcolor=lightyellow
      label="Top Floor"
      Bedroom
      Bathroom
    }
  }
}

\ssec Low-level TeX interface
\vfil

\def\forpy{\textfile{for.py}{for i in range(10): print(i^2,end=" ")}}

\guix{py}{\forpy}
\guix{pyresult}{\eval{python/python3}{\forpy}}

\hisyntax{python}
\verbinput (-) \guixref{py}

\verbinput (-) \guixref{pyresult}

\bigskip
\hrule
\bigskip

\def\largel{\textfile{largel.lisp}{%
(defun myfunc (x)
  (loop for i to x
    do (format t "Hello ~a " i)))
(myfunc 5)
}}

\guix{lis}{\largel}

\guix{lispy}{\eval{sbcl --script}{\largel}}

\verbinput (-) \guixref{lis}

\verbinput (-) \guixref{lispy}

\vfil

\pg.
